\documentclass{article}
\usepackage{pdfpages}
\usepackage{amsmath}
\title{Ph20.7 - Encryption}
\author{Stella Wang}
\date{16 March 2018}
\begin{document}
	\maketitle
\section{Threat Model}
A group of anonymous hackers plan to infiltrate CNN. They aren't planning to do anything with the data they hack, they only want to change the logo to FNN while they have control. Even though their plans are politically motivated, they have no government backing, or really great hackers on their side. That being said, we have reason to believe that the threat is credible because they have the ability to monitor network traffic and are based in Atlanta, where the CNN headquarters are. If they are unable to gain remote access, they will use physical force to gain access a CNN computer. 
\section{Diffie-Hellman}
Diffie-Hellman was one of the first public-key protocols. It differs for other traditional key exchange methods because those usually require that two parties exchange keys over a secure channel. The Diffie-Hellman method allows for key exchange over insecure channels by means of establishing a shared secret key. The Diffie-Hellman protocol is as follows: 
\begin{enumerate}
	\item Two parties agree on finite cyclic group and generating element (group theory)- an example is using primitive root modulo
	\item each party independently picks a random number and applies it to the generating element of the cyclic group. They send the result of applying their number to the generating element to each other, but not their number
	\item They are both able to compute a group element achieved by applying both numbers to the generating element. This is the shared secret key
\end{enumerate}
Because the cyclic group and generating element are agreed on before hand and presumably only known by the parties involved, it becomes computationally difficult to compute the secret key.  This process can also be generalized for more than 2 parties, where the order of the parties sending information to each other would need to be taken into account. The main weakness of the Diffie-Helman method is that it relies on choosing an appropriate cyclic group and generating element. In order for it to be secure, it is recommended that the group used be of order 2048 bits or greater. Unfortunately, most groups are not on that order, which makes it possible for hackers to use the number field algorithm, which significantly reduces the computation necessary to obtain a specific key. To prevent this type of attack, it is recommended that the group order is either increased or elliptic curve cryptography is used. 
\section{SSH}
Secure Shell (SSH) is used for securely operating network services over an unsecured network. It is commonly used for remote login to computer systems. SSH uses a public-key cryptography to authenticate and end-to-end encrytion for creating secure channels. SSH can be used for tunneling, forwarding TXP ports and fill transfer through the (SFTP) and (SCP) protocols. The architecture for SSH involves many well-separated layers which include a transport, authentication and connection layer.
There is indication that the NSA and CIA are able to decrypt and intercept SSH connections by installing software onto Windows or Linux operating systems.
\section{GPG}
GNU Privacy Guard is a free hybrid encryption software. It relies on a combination of conventional symmetric-key cryptography and public-key cryptography. It encrypts messages using asymmetric key pairs generated by users.
\section{Proof of Work}
-----BEGIN PGP PUBLIC KEY BLOCK-----

mQENBFqr4UEBCADOvjCbkDT1/m1USUYKuha3upbypdZR4/VIcgS2FPV3/+BdmvP7
3+pR0qyswKawgunEtOH36osfPAn8Vy3BclA1N+lyMk9HbgeujUhhNpOSDhx0U87g
p3aFxg/KE8PW7BT9goK0zqMPOIWXugEL64fiH2cSSAB7T38qqIwDJxcmyA0JFIw3
50eFX13NDPIuHNmgBOM2HNZ8w5YUIlDeW9pk6Fca6yqZL+bLQL/tgRHR9PHj+CRM
rywZpRW8pDTprrc2Ax/Q7m/yLCwMjOQQ0TEhYbaeq0SjOhPIeCm6ULc/Ml842Udd
n/nv0GXhDAGUCEjihlrvD9n4WBiogrvVrDdDABEBAAG0JlN0ZWxsYSBXYW5nIDxz
dGVsbGEud2FuZzM5OEBnbWFpbC5jb20+iQFUBBMBCAA+FiEEbaFXLn2JWVemSWMZ
Tq4QNqGHIE8FAlqr4UECGwMFCQPCZwAFCwkIBwIGFQgJCgsCBBYCAwECHgECF4AA
CgkQTq4QNqGHIE8WXgf8DiQbLEnFdnkEJlrxEN+3nj3c+CcvUf3+Zo8yfL5GW4ky
/Obx3yyLd14EGX6SReHUAs07gTjfdgok+RNwsv2CTH6dRyp7LJw3/mw/EFGUUziw
MdrqmH60L88IhEHyJz9aEzS4KJ9FjSHJH6EwBR7Rk4kv6tCwErNPHNlPq82w67A+
tz3V1HZxdUG6YIISdnSYeTHE8fPKikS5RtS3jvHhKIC0eiI3rNfPL7hPbgfQpTMD
iGb9DnYsOtPH1o8AeiijR6CHi2qXl7sl5BYSTvfykGh2RCfKU3jtj8Yp6Ob/YmCK
mMlYTDMgP+VZWUqizqRDMjk7CLYyl59v9UkD3KmrzrkBDQRaq+FBAQgA0SKwKgEt
foxHvXui69qIz06IiU1IWRSjfbiew6WXGPIOZMfDdaGNA1nu6BWFYozvopmN7176
2UlaDssIqt6Y6maT5llDmpHoGavoMVe2GRcDhO2950GUMWF1mxgG1KMxOAA3i9d2
fdHMUMPlAu3NPEJECFukuMSXSUpRSThoAn3afx7rY0llCTa5uPvgtV5qsRXIZl9e
5RS4cRpU7qh0Porg5yCwhxxOuRtv5LmT92uQauSbSPEops/71eFNV+6xQRdt/BUj
cC48qDFNDQkcZrfKMTZpRPZ1PcDAZ9um/cYHdswAqi0/EiLKqhHWdekK+mWZINAt
8EcCoUWhpAWTewARAQABiQE8BBgBCAAmFiEEbaFXLn2JWVemSWMZTq4QNqGHIE8F
Alqr4UECGwwFCQPCZwAACgkQTq4QNqGHIE+ntQgApjlSMPCn8Shy5t3T5CM+g7q/
CbT9O+7rgMxZHj5cjxG2kw8AlWGNvb8F6pQo5rJnsuaUfsgm7RsDFOY0GIMxxk7t
DyL8+CCCWBF4O6Tmp5cvIpyYUzqWxXSqzbzrR9hapHmfq4/9dQM0n8xC3+L7PmBN
6UdKQausoRXMVaj5kUqohjh7xVne5Fa9pPgFWM6rKf5OT4CWk4onQOtVTcp4k7JO
LxY134+3EVv8nEsc86Ed5XfhKwobWlsvJyRFQpHzOnGk3EU9LA63JnxJ8pVaQ0HN
PaSYpJUz1Fph+3sI7I2i85wAFfSxnxHFyVAMfn7IjIm//fZzCkMNcXm4hg1DEQ==
=5nc6
-----END PGP PUBLIC KEY BLOCK-----
\section{Summary}
A Secure Authentication mechanism involves establishing the identity of a client. This involves use of a registry which stores user information and credentials. When a client wishes to gain access, the credentials they give must match those in the authentication mechanisms repository.
Strong passwords can be compromised through key logger, phishing and shoulder surfing attacks. These attacks relate to password usage and rely on the missharing of passwords, which make even Strong passwords susceptible.

I have learnt more about the Diffie-Helman protocol, SSH and gpg through this assignment. Using this knowledge I will be able to access files and communicate with other people more securely before we establish a plan to combat the threat. 
\end{document}